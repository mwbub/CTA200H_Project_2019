\documentclass[10pt]{article}
\usepackage[utf8]{inputenc}
\usepackage[margin=3cm]{geometry}
\usepackage{amsmath}
\usepackage[style=authoryear, citestyle=authoryear]{biblatex}
\usepackage[colorlinks=true, citecolor=blue, linkcolor=blue, hyperfootnotes=false]{hyperref} 
\usepackage[multiple]{footmisc}

\title{CTA200H Computing Assignment}
\author{Mathew Bub \\ Supervisor: Cristobal Petrovich}
\date{May 17, 2019}

\newcommand{\evec}{\mathbf{e}}
\newcommand{\ehat}{\hat{\evec}}
\newcommand{\jvec}{\mathbf{j}}
\newcommand{\jhat}{\hat{\jvec}}
\newcommand{\Jvec}{\mathbf{J}}
\newcommand{\rpl}{\mathbf{r_{\mathrm{pl}}}}
\newcommand{\rplhat}{\mathbf{\hat{r}_{\mathrm{pl}}}}
\newcommand{\xvec}{\mathbf{x}}

\addbibresource{refs.bib}

\begin{document}

\maketitle

\section*{Introduction}
This assignment studies the evolution of a binary system under the influence of the tidal field of a distant third body. In this case, we study the evolution of the orbit of the Moon under the effect of the tidal field of the Sun.

Here, we will use vectorial formalism \parencite{tremaine2014} to describe the orbits under study, rather than the more traditional orbital elements. Under the vectorial formalism, we define an orthonormal basis $\{\ehat, \jhat \times \ehat, \jhat\}$, where $\ehat$ points in the direction of the pericentre of the Moon's orbit, and $\jhat$ points in the direction of the angular momentum $\Jvec$. Then, the shape of the Moon's orbit is described by
\begin{align}
    \evec &= e \, \ehat \label{eq:evec} \\
    \jvec &= \sqrt{1-e^2} \, \jhat \label{eq:jvec}
\end{align}
where $e$ is the eccentricity of the orbit. 

\section*{Part 1: The Evolution of Moons Around Planets}
Assuming that the Moon is massless, then according to the assignment sheet and the accompanying notes by Antognini, the secular evolution for the Moon perturbed by the Sun reads
\begin{align}
    \frac{d\evec}{dt} &= -\tau_{\mathrm{moon}}^{-1}[2 \jvec \times \evec - 5(\rplhat \cdot \evec) \jvec \times \rplhat + (\rplhat \cdot \jvec) \evec \times \rplhat] \label{eq:devec} \\
    \frac{d\jvec}{dt} &= -\tau_{\mathrm{moon}}^{-1}[(\rplhat \cdot \jvec) \jvec \times \rplhat - 5(\rplhat \cdot \evec) \evec \times \rplhat] \label{eq:djvec}
\end{align}
where $\rpl$ is the position of the Earth in its orbit about the Sun, which is assumed to be circular and is therefore given by
\begin{equation} \label{eq:rpl}
    \rpl(t) = 1 \, \mathrm{AU} \, (\cos(2 \pi t/1 \, \mathrm{year}), \; \sin(2 \pi t/1 \, \mathrm{year}), \; 0)
\end{equation}
and
\begin{equation} \label{eq:tau}
    \tau_{\mathrm{moon}} = \frac{m_{\mathrm{earth}}}{m_{\mathrm{sun}}} \frac{r_{\mathrm{pl}}(t)^3}{a_{\mathrm{moon}}^3} \frac{P_{\mathrm{moon}}}{3 \pi}.
\end{equation}
By Kepler's third law, we also have that
\begin{equation} \label{eq:Pmoon}
    P_{\mathrm{moon}} = 2\pi \left( \frac{a_{\mathrm{moon}}^3}{Gm_{\mathrm{earth}}} \right)^{1/2}.
\end{equation}

Using this, we can calculate $\tau_{\mathrm{moon}}$ for the current separation of the Moon. Working in units of AU/$M_\odot$/yr, we have that\footnote{\url{https://en.wikipedia.org/wiki/Gravitational\_constant}}\footnote{\url{https://en.wikipedia.org/wiki/Earth\_mass}}\footnote{\url{https://en.wikipedia.org/wiki/Lunar\_distance\_(astronomy)}}
\begin{align*}
    G &\approx 4 \pi^2 \; \mathrm{AU}^3 \: \mathrm{yr}^{-2} \: M_\odot^{-1} \\
    m_{\mathrm{earth}} &= 3.003 \times 10^{-6} \; M_\odot \\
    a_{\mathrm{moon}} &= 1/388.6 \; \mathrm{AU}.
\end{align*}
Substituting these into (\ref{eq:tau}) gives (see the accompanying Python script)
\[
    \tau_{\mathrm{moon}} = 1.409 \; \mathrm{yr}.
\]

We now implement a $4^\mathrm{th}$ order Runge-Kutta (RK4) integrator to evolve equations (\ref{eq:devec}) and (\ref{eq:djvec}). In general, given a system of ordinary differential equations $\xvec' = \mathbf{F}(\xvec, t)$ and initial data $\xvec(t)$, we can compute $\xvec(t + \Delta t)$ using the RK4 method according to the formula
\begin{equation} \label{eq:RK4}
    \xvec(t + \Delta t) = \xvec(t) + \frac{\mathbf{m} + 2\mathbf{n} + 2\mathbf{p} + \mathbf{q}}{6} \Delta t
\end{equation}
where
\begin{align}
\begin{split} \label{eq:RK4_derivatives}
    \mathbf{m} &= \mathbf{F} \left( \xvec, t \right) \\
    \mathbf{n} &= \mathbf{F} \left( \xvec + \textbf{m} \cdot \tfrac{1}{2} \Delta t, t + \tfrac{1}{2} \Delta t \right) \\
    \mathbf{p} &= \mathbf{F} \left( \xvec + \textbf{n} \cdot \tfrac{1}{2} \Delta t, t + \tfrac{1}{2} \Delta t \right) \\
    \mathbf{q} &= \mathbf{F} \left( \xvec + \textbf{p} \cdot \Delta t, t + \Delta t \right)
\end{split}
\end{align}
\parencite{hirsch2013}.

\printbibliography
\end{document}
